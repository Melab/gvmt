

GVMT is, as the name suggests, supplies a number of tools. Six of these form the core of the toolkit. 

As mentioned in the previous chapter, the GVMT tools can be grouped into front-end and back-end tools. The front-end tools convert source code into code for the GVMT Abstract Machine. Back-end tools convert the abstract representation into real manchine code.

\section {Core tools}
The six core tools are:
\begin{itemize}
 \item Front end tools
\begin{itemize}
\item \gvmtic
\item \gvmtc
\item \gvmtxc
\end{itemize}
 \item Back end tools
\begin{itemize}
\item \gvmtas
\item \gvmtcc
\item \glink
\end{itemize}
\end{itemize}

\gvmtic{} and \gvmtc{} form the front-end and are used to translate the C-based interpreter description and standard C files, respectively, into GSC format.

\gvmtas{} converts these GSC files to object files and \glink{} links them together.
\gvmtcc{} takes the GSC file produced by \gvmtic{} and produces a compiler from it.

\subsection{Interpreter description file}
An interpreter description file consists of bytecode definitions plus the local variables shared by all bytecodes.

Each bytecode can be defined by its stack effect plus a snippet of C code or as a sequence of other bytecodes and predefined GVMT abstract stack machine codes.

Usually there is one primary interpreter description file that defines the bytecode format. Secondary interpreters (see Sect. \ref{sect:gvmtxc} will be guaranteed to use the same bytecode format. However it is possible to have several independent primary interpreters. For example a virtual machine might use one bytecode format for its interpreter and compiler and an entirely different format for its regular-expression engine. Each primary interpreter may have secondary interpreters.

Interpreters are fully re-entrant and thread-safe. The interpreter description file  defines not only the behaviour of the interpreter, but the behaviour of the GVMT-generated compiler as well.

\subsubsection{Format}

The interpreter description file consists of a locals block and any number of bytecode declarations in any order. 
\subsubsection{Interpreter-local variables}
Interpreter-locals variables are declared in a locals section:
\begin{verbatim}
locals {
        type name;
\end{verbatim}
\nopagebreak
        \vdots
\begin{verbatim}
}
\end{verbatim}

Interpreter-local variables are visible within all bytecode definitions, and can be accessed indirectly via the interpreter frame on the control stack. Interpreter-local variables (and the interpreter frame) have the same lifetime as the invocation of the interpreter.

\subsubsection{Bytecode definitions}
Bytecodes can be defined either in C or directly in GSC. Both types of definition can be used in the same interpreter.

C-bytecode definitions are in the form

\begin{verbatim}
name (`[' qualifier* `]')?  `(' stack_comment `)' [ `=' <int> ] `{'
    Arbitrary C code
`}'
\end{verbatim}

Legal qualifiers are:
\begin{description}
\item [private] This bytecode is not visible at the interpreter level.
\item [protected] This bytecode is acceptable to the interpreter, but not to any verification code. It is only to be used internally as an optimisation. The meaning of `protected' is largely conventional. GVMT will ignore it.
\item [nocomp] This bytecode will not appear in bytecode passed to the compiler. If this bytecode is seen by the compiler it will abort.
Useful for reducing the size of the compiler.
\item [componly] This bytecode will only be passed to the compiler. It will cause the interpreter to abort. Useful for reducing the size of the interpreter.
\end{description}

The \verb|stack_comment| is of the form:
 \begin{verbatim}
type id (`,' type, id)* -- type id (`,' type, id)*)
\end{verbatim}

Where \verb|type| is any legal C type and \verb|id| is any legal C identifier, or a legal C identifier prefixed with \verb|#|. Variables prefixed with \verb|#| are fetched from the instruction stream, not popped from the stack. When referred to in the C code, the \verb|#| prefix is dropped.

The integer at the end of the header line is the actual numerical value of this bytecode. If not supplied, GVMT will allocate bytecode numbers starting from 1.

Compound instructions are in the form:
\begin{verbatim}
name (`[' qualifier* `]')?  '(' stack comment ')' [ `=' <int> ] `:'
   instruction*
`;'
\end{verbatim}

Where \verb|instruction| is a legal GSC or user-defined instruction.

To allow user defined instructions to do anything useful, a large number of builtin instructions are provided. All are private, that is they must be used inside a compound instruction to be visible at the interpreter level. 

Instruction stream manipulation operators are provided:
\verb|#@| Removes a byte from the instructino stream and pushes it to the data stack.
\verb|#N| where N is a one-byte integers pushes N into the front of the instruction stream.
\verb|#[N]| Pushes a copy of the N\textsuperscript{th} item, starting at zero, in the instruction stream into the front of the instruction stream.
\verb|#+| or \verb|#-| Adds or subtracts, respectively the first two values from the instruction stream  and pushes the result back to the front of the instruction stream.

Appendix \ref{app:inst} contains a full list of all abstract machine instructions 

There are four bytecode names which are treated specially by the GVMT, these are:
\begin{itemize}
\item \_\_preamble 
\item \_\_postamble
\item \_\_enter
\item \_\_exit
\end{itemize}
\verb|__preamble| is executed when the interpreter is called, before any bytecodes are executed.

\verb|__postamble| is executed when the interpreter reaches the end of the bytecodes. Interpreters with a \verb|__postamble| instruction expect a second parameter, the end of the bytecode sequence. This is useful for `interpreters' which analyse code rather than executing it.

\verb|__enter| is executed before the main bytecode definition for every bytecode executed.

\verb|__exit| is executed after the main bytecode definition for every bytecode executed.


\subsubsection*{Examples}


TO DO -- Put a couple of examples here. 1 C, 1 vmc.



\subsection{The interpreter generator \gvmtic{}\label{sect:gvmtic}}
\gvmtic{} comiles an interpreter description file, as described above, into a GSC file.

\gvmtic{} takes the following options:

\input{gvmtic_options.tex}

\subsection{The C compiler \gvmtc{}\label{sect:gvmtc}}
\gvmtc{} takes a standard C89 file as its input and outputs a GSC file.

\gvmtc{} takes the following options:

\input{gvmtc_options.tex}

\subsection{The GSC assembler \gvmtas{}\label{sect:gvmtas}}
\gvmtas{} takes a GSC file and assembles it to a GVMT object file (.gso is the expected file extension)

\gvmtas{} takes the following options:

\input{gvmtas_options.tex}

\subsection{The compiler generator \gvmtcc{}\label{sect:gvmtcc}}
\gvmtcc{} Takes a GSC file produced by \gvmtic{} and produces a compiler. The generated compiler is in C++ and relies on LLVM (http://llvm.org/).
It will need to be compiled with the system C++ compiler.

It may be possible to generate compilers using other code-generator back-ends in the future, but there is currently no plan to do so.

\gvmtcc{} takes the following options:

\input{gvmtcc_options.tex}

\subsection{The linker \glink{}\label{sect:glink}}
\glink{} takes any number of GVMT object (.gso) files and links them to form a single native object file.
This object file can be linked with the GC object file and compiler object file (if required)
using the system linker to form an executable.

\glink{} takes the following options:

\input{gvmtlink_options.tex}

\subsection{lcc-gvmt\label{sect:lcc}}
Both \gvmtic{} and \gvmtc{} use a modified version of lcc as their C-to-GSC compiler. This compiler uses the standard lcc front-end and a custom back-end. The back-end operates in four passes. The first pass labels the IR forests with the C type for that expression, as much as can be derived without the explicit casts present in the source code. The second pass is a bottom pass which labels the trees with the set 
of possible GSC types it may take. The third pass is a top-down pass which attempts to find the exact GSC type of each node. The fourth pass emits the GSC.

\section{Other tools}

\subsection{The bytecode-processor generator \gvmtxc\label{sect:gvmtxc}}
\gvmtxc\ Takes a (partial) interpreter description file and a master GSC file (produced by \gvmtic) and produces a GSC file representing an interpreter. This interpreter is guaranteed to accept the same bytecode format as the master interpreter.
For a partial interpreter description file, the generate interpreter skips over the omitted bytecodes.

\gvmtxc{} takes the following options:

\input{gvmtxc_options.tex}


\subsection{The object layout tool\label{sect:objects}}

The object layout tool is an optional tool to help manage heap object layout. It can create C structs, shape vectors and marshalling to GSC files for objects all from a single specification.
This tool is designed as an extensible Python script rather than a stand-alone tool, although it can be used as such.
When used as a standalone tool it has takes the following options:

\input{objects_options.tex}
